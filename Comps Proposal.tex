\documentclass[10pt,twocolumn]{article} 

\usepackage{oxycomps} % use the main oxycomps style file

\usepackage{biblatex}
\addbibresource{references.bib}


\pdfinfo{
    /Title (Comps Topic Proposal)
    /Author (Malcolm Holman)
}

\title{Comps Proposal}

\author{Malcolm Holman}
\affiliation{Occidental College}
\email{holmanm@oxy.edu}

\begin{document}

\maketitle

\section{Problem Context}

    The problem at hand is how nuanced training for indoor and outdoor rock climbing can be for different people. As a rock climber, I have faced frustrating plateaus that required more training to overcome than I knew how to implement at the time. Today, I have more knowledge about the basics, but I continue to train and would benefit from learning about new exercises This lack of knowledge is common for many beginner (and intermediate) climbers and is the first part of the problem I am trying to address. 

	Being a beginner rock climber is a rewarding experience because most low grade, easier climbs can eventually be worked out with decent strength and poor technique. This stage is often short lived as people begin to want to climb harder problems/routes. Past a certain threshold of difficulty, technique and strength begin to work together and brute force is not always an option. This is the first place many people begin to get frustrated but with some technique tips, one can get better without much strength training. During this stage, I often found myself watching videos and reading articles trying to piece together tips and tricks from different people across the internet. A climbing app could streamline this process by compiling some standard beginner technique mistakes and how to fix them. These tips can also extend beyond just beginner mistakes to provide intermediate and advanced climbers with more nuanced technique. 

	As climbers enter the intermediate stages, often what prevents people from progressing to harder grades is strength, not technique. While there is always room to improve technique, many intermediate climbers are capable of explaining and understanding the technique needed to do a harder problem but may not have the strength to do so. This problem can directly be solved with climbing specific strength training, and is the second issue addressed by the app. Strength training for climbing is particularly challenging because of how sensitive peoples skin, fingers and finger tendons, wrists, and elbows are to injury and tweaks. As a result, a training plan that works for some people may be too strenuous or aggravate a sensitive spot for others. 
	
	The strength training portion of the app will be focused on workout creation, customization, with the possibility of user interactions. The create feature would allow users to create their own workouts by choosing from a list of provided exercises as well as uploading custom exercises. User input custom exercises will allow for descriptions, photos, and videos to be uploaded along with them. Many existing climbing training apps can provide exercises and workouts but allow for little customization within the workouts given. This can be solved by having every provided workout allow for each exercise to be swapped individually with prompted alternative exercises that target the same muscle group or an exercise of the user’s choosing. Basic user interaction, such as following others and liking workouts, could be implemented because climbing is typically a social activity. This would create opportunities for people to share workouts that they have made with friends and fellow climbers who can then use or modify the workout.
	
\section{Prior Work}

    There are more plenty of climbing apps available and even more general fitness/workout apps but none have the same customization as the app being proposed. The most popular climbing training app on Apple’s iPhone app store (my target market) is Crimpd. Crimpd provides users with a handful of useful workouts as well as on-wall circuit training and paid professional training plan. The free workouts that are provided are curated by professional climbers so they are quite good, but to expand upon them you have to pay \$13/month, which is too much. The app also lacks customization of workouts and does not have a timer. 

	The closest apps to what I would like to build are Fitbod and Flex, both of which are general fitness apps, not specifically for climbing. Flex initially prompts the user for their goals and desired intensity and creates an initial workout based upon the input information. The best feature though is what they call “freestyle workouts” which allow the user to select from a large list of workouts that are in their database. The workouts can be broken down into sets, weight used, and duration. Fitbod is another app that I plan to draw inspiration from due to its modular workout design. It allows users to input the equipment that is available to them at their gym and recommends specific workouts accordingly. Following the creation of the workout users have the option to modify the amount of reps, duration, and weight of the exercise. There is also an option to swap the provided exercises for others in their database, similar to Flex.

	
\section{Evaluation and Methods}

	Evaluation of the project is complicated because it is tempting to want to measure its success based upon the results of climbers who train using the provided exercises which is unrealistic for several reasons. First, this would mean that the app would have to be deployed so that users can use it and document their training on their own time. While deployment is realistically possible, it would not occur until late in the project’s timeline leaving little room for progress assessment. Another limitation is the amount of data that would be asked of users without any incentive. Finally, the collection of data from multiple participants over time is too challenging for a team of one to collect in the provided time frame. So, instead of hard qualitative metrics, user interviews will be conducted to assess the effectiveness, design quality, and issues with the app. 

\subsection{Background}

	Interviews can be broken down into 3 distinct types: structured, unstructured, and semi structured. Structured interviews use a fixed series of close ended questions that is the same for every user. Structured interviews are useful because they are quick, allow for more participants, and give more concrete results for analysis. The drawback of structured interviews is that they do not allow for users to elaborate on their ideas. As such, they are more useful towards the end of a project because they can help to assess if goals were accomplished and not much else.  Unstructured interviews lack guided questions and instead flow with the user. They are useful because they accumulate the most possible qualitative data from the users, find pain points and problems, and provide solutions and suggestions. Unstructured interviews need to be carefully conducted because the lack of control over information obtained can create user bias.

\subsection{Interviews}

	For this project, the first round of interviews will be conducted during early stages of production, ideally within the first weeks of the school year. This will allow for the target demographic to have early input about what they would want to see in the app’s final form. Early open ended interviews will also help to identify any glaring holes in the project. These interviews will be less about measuring or assessing the success of the project and more about the direction that it should be headed. For example, users will be presented with a preliminary diagrams for the pages of the app and asked how they would navigate. 
	
    About three quarters of the way through the semester, ideally sometime in late October, a second round of user interviews will be conducted. The second round of interviews will likely be a combination of structured and semi structured interviews. There will be a set of questions that will be asked every time with room for elaboration on those questions. This combination will help to provide late stage feedback which will be useful to assess the success of the project. CITE OpenXCell
    
    Following the collection of qualitative data from the first and second rounds of interviews, an analysis will take place. Throughout the interview process, ample notes will be taken in addition to a recording of the conversation, provided user consent is obtained. One method of analysis I plan to use is thematic analysis. Thematic analysis consists of: reviewing the notes and the recording, summarizing key repeated points and giving them codes, categorizing the codes, grouping the codes according to similarities, and finally using the categorized groups to find specific themes. These themes could range from criticism of pain points to useful suggestions and will be useful for both the first and second stage of interviews. \cite{christian_user_2021}
    
\subsection{Presentation}
    
    After all of the exercises for the app are collected and inserted into a database, they have to be presented to the users. Users will be given the ability to filter exercises and add them to a custom workout that they can save. This will update a database and allow for each user to have their own collection of workouts. In order to use the app however, users must first complete a brief education on the risks and benefits of climbing training. This education will be accompanied with a collection of injury prevention exercises and stretches.  
	
\section{Technical Background}

    I have considered algorithmically generating workouts for people based on input criteria but I think that there are ethical concerns associated with the users risk of injury. It is a feature that could be implemented in the future. Without algorithmic workout generation, there is not much technically needed beyond mobile app development. I have worked with web apps in the past and I understand the basic workflow. For this project, since it is a mobile app not a web app, I will have to learn new things. I will be using react native and Firebase to build a mobile app that will be optimized for iPhone.

    
    I chose Firebase because it is a free database service that should be able to handle the amount of reading and writing that will occur while developing my project with the option to pay for more in the future. There is also ample documentation making it easier to pick up. I chose react because it is faster and more streamlined than vanilla JavaScript. 
    The bulk of the project comes down to the exercises chosen and the information provided to the user about the exercises and their application. Including a wide variety of exercises that can accommodate different skill levels and workout environments is essential for the effectiveness of the project. Essentially, a large collection of selected exercises compiled from climbing websites, along with their target muscle groups and brief descriptions will need to be entered into a database and presented to users. In addition to giving users an abundance of exercises, they must first be given ample information and exercises to prevent any injuries. Additionally, for each exercise, instructions and tips will be provided. The training exercises can be broken down into 3 sub groups: home fitness and strength training, endurance training (in a gym), and strength training (in a gym). 

\subsection{Beginner Technique Tips}

    Providing and promoting tips for good technique will improve the success of the app because it will make it more useful for beginners. The best way to train for a beginner is simply to climb a couple times per week with good technique. Strength training can assist in a user's progress but the fundamentals of technique and basic movement are essential. 

	The goal of climbing technique is to make routes as easy as possible by manipulating one’s body into advantageous positions. The primary way that technique can be improved is through focused footwork. Good foot technique can take the burden off of small finger and forearm muscles and share it with larger, stronger leg muscles. Using the edge of climbing shoes is advantageous because it creates more contact and stability with the hold. Smearing is using the wall as a foothold by pressing the tip of a climbing shoe into it, and is useful when there are no footholds. Keeping a low heel increases contact with holds and makes them feel more comfortable, especially when smearing. Finally, looking for footholds before handholds will reduce the strain put onto a climber's hands when they decide to make a move. 
	Footwork is not the only way to make climbing easier, there are upper body techniques that can be put in place to reduce the stress put onto muscles. Climbing with straight arms as much as possible will place more weight on a climber’s skeleton than their muscles. Over time this will reduce fatigue and make harder climbs easier. One way to make climbing with straight arms easier is by keeping at least one hip close and leaning one’s upper body back. Finally, climbing with opposite hands and feet keeps climbers more stable and avoids what climbers call a “barn door” — when a climber’s hands and feet are on the same side and the climber swings off the wall. \cite{rei_climbing_nodate}


\subsection{Injury Prevention}
    
    Ample information about the risks of climbing training and exercises that help to prevent injury are going to be introduced to the user before anything else. As is expressed in section (ethics section), climbing is a particularly dangerous sport to train, so educating users on exercises that aid injury prevention is crucial. The app is not intended to help climbers recover from injuries because that can require professional assessment and physical therapy. The following exercises are intended to be used by users who are not currently injured, rather they are preventative. 

    Finger, hand, and forearm injury prevention exercises are a primary emphasis due to the increased stress placed on the muscles and tendons when climbing. Finger and hand rollouts can be used to release tension and increase circulation to hand and thumb muscles. They are done by placing a small massage ball or rubber band ball on a flat surface and rolling out one’s palms and front of fingers for 30-60 seconds per hand. Finger tendon glides move finger flexors through their full range of motion and increase mobility and circulation. They consist of a three step process that is repeated in reverse to complete one rep. First press fingers flat against palm, then curl them into a fist, then unroll join by joint and reverse the process, repeated in 3 sets of 8-12. The forearm massage is simple but effective as it releases tension and increases circulation in forearms. The execution is not too complicated, use a massage ball (side of a can, a massage stick behind a knee, a massage gun, lacrosse ball, or an ArmAid will suffice) to gently massage forearms and back of forearms, avoiding the wrist area and bony part of elbows. \cite{carpenter_prevent_2020} 
    
    The namaste stretch is a three part exercise that stretches wrist and finger flexor muscles in 3 directions, one with each part. Of the exercises mentioned thus far it has the most risk for over stretching, so users should be reminded to stop when their fingers bow, palms pull apart, or stretch is too painful. First, place palms together at shoulder height with palms together and at shoulder height with fingers facing towards one’s chin and elbows towards the side. With relaxed shoulders, slowly bring hands down until a stretch is felt in both wrists and forearms. Once a stretch is felt, pause and spread fingers apart, then pause again and reverse the movement to the starting position, repeating twice. The second part is similar but arms are extended in front at chest level with fingers facing forward. The movement is the same but towards the user’s chest. The third step repeats the same process but with arms extended down towards the ground. \cite{carpenter_prevent_2020} 
    
    Shoulder, back, and hip injury prevention is also focused, but not as emphasized as wrists, arms and fingers. Hip lunges with arm wheels combine all three into one stretch by stretching hip flexors and activation shoulders and upper back. They are done by entering a lunge with a straight back leg, sinking hips down until a stretch is felt in the hip flexor and quad of the back leg. Then there is a three step arm circle, slowly sweep arms overhead while keeping them straight, slowly pull shoulders back as elbows are bent to 90º, straighten arms as they come forward. They should be done with legs engaged, a relatively straight lower back, and in a smooth continuous motion. Resistance band reverse flys are used to strengthen the rotator cuff and scapular muscles. For this exercise, a long resistance band is recommended but can be replaced with yoga pants, running tights, or anything long and stretchy. There are three main ways to grip the band, with bent elbows, straight arms, or diagonally all of which pause at the end of the exercises and slowly reverse the movement to complete one rep. With bent elbows, grip the band with palms facing up and hands at shoulder width, a wide grip makes it easier and narrow makes it harder. Rotate arms outwards while keeping elbows in and arms at 90º. With straight arms, extend arms at chest level with palms up and a slight bend in the elbow. Then rotate arms out to the side while keeping shoulders relaxed and bringing shoulder blades together. With arms straight down reach one arm up and to the side keeping thumb facing up, while keeping the other arm down and opposite. \cite{carpenter_prevent_2020}
    
    Typically, it is recommended that more dynamic exercises such as finger tendon glides and resistance band flys be used before climbing to warm muscles up. Static exercises such as the namaste stretch are more applicable to off days from climbing or after training.
\subsection{Gym Training} 

\subsection{Overview}
    The primary exercises that climbers use to train in gym settings are pull ups and hangboarding. Users will benefit from being able to document their progress in both hangboarding and pull up training. Both exercises continue to scale with the skill of the climber, hanging on smaller edges or doing a pull up with more weight are examples of ways to modify the exercises. There is not a fixed hangboard or pull up routine that is effective for all climbers so providing users with a routine that works for them is going to be rather difficult.Instead, ample information about the benefits and risks of hangboarding and pull ups as well as proper technique and the ability to track progress for each exercise will be provided. With either form of training, users need to remember that climbing itself also puts a strain on the muscles being trained and resting and taking breaks when muscles feel overworked is always recommended.  

\subsection{Hangboarding}

	Hangboarding is known as one of the best ways to improve finger strength. Due to its intense strain on finger tendons, it is not recommended for beginner climbers. Even advanced climbers need to be careful when training using a hangboard as the injury risk is high. Proper form is essential to maximize the benefits of the training and mitigate injury. Climbers should use an open handed grip, meaning they should not wrap their thumb over for a full crimp or bend their fingers for a half crimp. Ample warm up is also recommended, this can consist of pull ups on jugs or a pull up bar or light low intensity climbing. Additionally, climbers should avoid locked elbows, keep their shoulder blades engaged, and keep their shoulders away from their ears. \cite{rei_how_nodate}

    Most commercial climbing gyms will have at least one brand of hangboard available for climbers to train on, however home modifications are possible and discussed in SECTION. Hangboarding consists of going into a dead hang on small holds for a short period of time. The hangboard should be at a comfortable height, meaning the user should be able to lower themselves into the tension on their fingers and put their feet down to release the tension. If the hangboard at the gym does not allow for this, users can use boxes or a bench to bring their feet closer to the board. When beginning hangboarding training, users should be using the larger holds on the board and slowly decrease the size of the holds over the course of several weeks. Again, the exact timeline varies from person to person and users should be reminded to listen to their body and work within their current limits. \cite{sheldon_how_2022} 

\subsection{Pull-ups}
    The pull up is the other key exercise for climbing training and it is less dangerous than hangboarding. While pull-ups are safer than hangboarding, they still are not fully recommended for beginners but not only because of injury risk. If beginner climbers train pull ups too soon, they may bypass the need for fundamental technique that aids in more advanced climbing when brute force is no longer an option. \cite{gripped_pull-ups_2021} Nonetheless, the pull up can be catered to climbers of all skill levels with certain modifications. Weaker climbers can use a weight assist in the form of a resistance band connected to the bar, a setup pulley system, or an assisted pull-up machine. More advanced climbers can hang additional weight from their waist to increase the difficulty as their strength increases. \cite{sheldon_how_2022} 


\subsection{Home Training}

	There is an entire field of climbing training dedicated to accessible exercises that can be done with little to no additional equipment. Many of these exercises have recently risen in popularity due to the coronavirus pandemic forcing even the most elite rock climbers to train at home. 
	To build finger strength at home, the primary recommendation is the utilization of door frames. A sturdy door frame can serve as both a hangboard or a pull up bar. Door frame edges tend to be on the smaller side, so for all of the following exercises the door frame can be replaced with a larger ledge or a tree branch if necessary to accommodate weaker users. Doorframe finger hangs are trained in the same way as traditional hangboarding, mentioned in the gym strength training section, with the same words of caution: for intermediate and advanced climbers only, requires warm up, use an open crimp, and do not wrap your thumb over. Finger pull ups target both finger and pulling strength and can be done in door frames with the same rules of caution. Hanging leg raises target finger and core strength and can be done on the same door frame with the same level of caution. The goal of leg raises is to dead hang from the frame and bring your legs as high as possible, bending knees to decrease difficulty. \cite{ward_training_2016}
	
	Home exercises are not limited to door frames though, as any open space can be used to strengthen target muscle groups. Pistol squats are useful for leg strength, stability and flexibility. Pistol squats are difficult exercises, they can be modified by grabbing on to something when standing up, or doing a 2 legged body squat instead. If a body squat is too easy and a pistol squat is too hard, users can add weight to their body squats to build strength. Another leg exercise is lunges, useful for strength, stability, and mobility. Lunges have many variations as they can be executed statically, while walking, or while jumping. The difficulty of each style depends on the user but all can be made more challenging by safely adding weight over time. Stair walks are another flexible leg strengthening exercise that can be done almost anywhere. Stair walks done with lower weight should go for more speed to improve general fitness and cardio. When executed with higher weight, (heavy objects, water jugs etc.) users should focus on stability and strength while walking. Finally, stemming burnouts are an excellent climbing specific leg focused exercise that can be done anywhere there is a narrow hallway users do not mind smudging. Stemming is when climbers use oppositional force in their arms and legs to suspend themselves. To train stemming, users should suspend themselves from their feet, staying upright and leaning left and right to engage one leg more than the other. \cite{ward_training_2016} 
\subsection{Endurance training}

	For the average gym climber, endurance training may be unnecessary and training attention would be better focused elsewhere. However, this does not mean that everybody cannot benefit from some form of endurance training. Climbing endurance training is different from typical endurance training for activities such as running or swimming. Climbing differs from these sports because it is not constant sustained muscle output, it is bursts or sequences of hard moves followed by some rest. Steve Bechtel, a rock climber and coach with several first ascents worldwide, wrote an article on www.climbstrong.com about endurance training. He breaks the training down into 5 key categories: strength and power, power-endurance, intensive endurance, extensive endurance, and stamina. For the sake of this project, only extensive, intensive, and power endurance will be included. \cite{bechtel_endurance_2017}

\subsection{Extensive Endurance}

	Extensive endurance is most useful for longer full pitch routes, trad climbing, and outdoor climbing in general. Extensive endurance training is less useful for indoor boulderers, as most indoor bouldering walls do not exceed 15 feet in height. The training is focused on being able to withstand low amounts of muscular stress for long periods of time. The intensity of this training is akin to that of  a warmup, so never reaching a point of failure. The training is intervallic and Bechtel recommends starting with 5 minutes on the wall followed by 5 minutes of rest. As strength and endurance builds, climbers can progress up to 30 minutes on the wall at a time. Because this climbing is easier and requires less effort, it also does not demand as much focus from the climber. This should be avoided because climbing with poor, unfocused technique builds inefficient muscle memory. \cite{bechtel_endurance_2017}

\subsection{Intensive Endurance}

	Intensive endurance, according to Bechtel, is the most important area of endurance training to target. It is most useful for hard medium length route climbing but can be applicable to longer boulder problems as well. The difficulty is supposed to feel a step harder than extensive endurance with shorter intervals. The interval splits are shorter and closer together for intensive endurance, Bechtel recommends 9 minutes of continuous climbing with a problem every 90 seconds repeated for multiple sets. The 6 problems should be somewhat difficult but climbers should not be completely exhausted at the end. Climbers should start this training targeting easier climbs while they figure out what climbs fall into this level of difficulty, gradually including harder climbs. Between sets, it is recommended that climbers take 10-15 minutes of rest to ensure that the next set can be executed with fresh muscles. \cite{bechtel_endurance_2017} 

\subsection{Power Endurance}

	Power endurance is a more complicated form of endurance because it can be broken down into aerobic power and anaerobic capacity. There is more risk associated with it because the increased intensity leads to a larger risk for injury. Generally it is not good to train power endurance more than 2 times per week, any more would require a trainer or a coach to ensure safety. Power endurance training should leave climbers feeling pumped and exhausted by the end of their session.

	Aerobic power training is the more sustained strength of the two, more commonly applied to hard route climbing. A recommended way to train aerobic power is doing 5 sets of 2 reps, with each rep being a double lap of a route. Rest between reps is recommended to be 1 minute, with 10 minutes in between sets to ensure a fresh attempt. The grade of aerobic power training should be 1-2 grades below the climber's consistent onsight grade, which means that by the second rep climbers are pushing their limits. Climbers should be pumped throughout and failing while training aerobic power should happen sometimes. \cite{randall_training_2019} 

	Anaerobic capacity training is focused on high intensity low volume training, more commonly found in bouldering. The recommended split is 6 sets of 3 reps with each rep being a hard, flash level boulder problem. The problems should be hard enough that the moves are not possible to maintain for more than 30-45 seconds. By the end of anaerobic capacity training, climbers should feel powered out and unable to climb anything else. Finally, the chosen problems should be steep and powerful, this training is less effective when climbing. \cite{randall_training_2019}
	
\section{Ethical Considerations}

\subsection{Safety Background}
    This project has an abundance of safety concerns for the users because not enough is being done to ensure climber safety while climbing and training. Climbing is a dangerous sport and due to the culture, muscles used, and intensity, training can also often lead to injury.  
	
	The risks associated with climbing and training are largely unknown to beginners and often ignored by more experienced climbers. While the risk of an acute serious injury or death is quite low, somewhere between .2 and 3.2 cases per 1000 hours, less acute but potentially long term recurring injuries are more common. As rock climbing has become more popular with the emergence of accessible indoor climbing options, overuse and chronic injuries from climbing have increased to 65\%, with up to 90\% of these injuries being to upper extremities. The most common of injuries to these upper extremities are injuries to the fingers and elbows. \cite{meyers_rock_nodate}

    Over 50\% of climbers have documented pain in their fingers, specifically in the distal interphalangeal and proximal interphalangeal joints that can be found in the index and long fingers (cite). \cite{meyers_rock_nodate} While these are the most common places for finger injury, other injuries such as injuries to the A2 pulley of the flexor sheath, and forearm tendons and ligaments are also common. Human fingers are not designed to bear the amount of weight that is required by some climbing holds which results in an imbalance in flexor and extensor forces, causing pain. In the elbow, pain is typically found in the medial epicondyle, lateral epicondyle and the anterior elbow, pain that is commonly referred to as “climbers elbow.” Intensity of elbow injuries can range from mild tendonitis and discomfort to bicep tendon ruptures in cases of extreme overuse. \cite{meyers_rock_nodate}

\subsection{Causes}

	The cause of these injuries is typically due to overuse, experience or lack thereof, and the demanding physical requirements of the sport and its training. Overuse comes in the form of overtraining syndrome (OTS), which as the name implies stems from repetitive training of the same muscle groups. The primary way to train for climbing is by climbing, but this project also promotes an abundance of off wall training that replicate movements that can be found on the wall and use the same muscle groups. Exercises such as hangboarding, pull ups, lock offs, and dead hangs put increased stress on the aforementioned injury prone areas and can be dangerous. The role of experience in relation to injury is rather complicated because these injuries affect both beginners and advanced climbers. Beginners have less knowledge about their body’s physical limits and while the climbs they are attempting put less stress on the fingers, their fingers and tendons will not be strong enough to withstand an abundance of off-wall training. Advanced climbers have stronger tendons and can handle more training but will end up facing the same challenges that beginners do on a harder scale. As climbs become more difficult, the wall gets steeper, the holds get smaller, and more weight is demanded to be put on upper extremities, increasing the risk of injury. This increase in climb difficulty also demands a more intense off wall training routine to maintain and increase strength. \cite{meyers_rock_nodate}

\subsection{Prevention}

    Presenting users with information about injury prevention is imperative. Before engaging in a custom climbing training regiment, users need the necessary knowledge to ensure they are as risk averse as possible. Even small injuries and lower amounts of pain need to be addressed early, or prevented, so that they do not deteriorate into chronic long term issues. There are four main categories of climbing injury prevention: proper warmup, antagonistic training, stretching, and rest. 

    The first effective method of injury prevention is ample dynamic warm up. Studies have shown that static stretching before an activity increases the risk of injury, but a targeted dynamic warmup increases blood flow to muscles so that they can safely perform. \cite{ellis_top_nodate} Users need to be presented with an abundance of exercises to warm up their muscles, especially their upper extremities, to decrease the risk of injury. The second method of recommended injury prevention is the training of antagonistic muscles. Climbing puts stress on the same muscle groups throughout both climbing and training which creates strength imbalances that put unnecessary strain on tendons and ligaments. Providing users with antagonist exercises and information about the muscles that they target will help longevity and injury prevention. \cite{meyers_rock_nodate} Third, users should be made aware of the value of stretching because when used properly it enhances muscle mobility and range of motion, reducing the risk of injury. \cite{ellis_top_nodate} Finally, and most importantly, users need to be made aware of the importance of rest. Sufficient rest is a crucial, often overlooked, step for muscle recovery and in turn injury prevention. 

\subsection{Accessibility}
    
    The proposed app clearly does not take into account potential users with disabilities. This issue is multifaceted because climbing as a sport requires vision, full body strength, focus, and coordination to be properly executed. While there are many potential groups who will face accessibility issues when using the proposed app, many of the more common accessibility concerns can be solved with enough setting customization. Examples of these settings include, but are not limited to: text to speech, text size customization, closed captions, mono audio, sticky keys, speech recognition, color blind options. \cite{gcf_computer_nodate}

    The easiest group to account for is those who are blind or have low vision (BLV). The term “easiest” is used because BLV has the most research done about it and as a result the most solutions. \cite{mack_what_2021} Granting accessibility to blind individuals is commonly done by providing users with text to speech and speech recognition functionality. This will allow blind users to interact with the app’s functionality without needing to see and tap the screen. For low vision and colorblind people, having alternate colorblind-friendly color schemes prepared will make their app experience smoother. For those with hearing impairments, any video instructions for exercises should have customizable closed captions along with them.

    Beyond general app accessibility, there are also climbing-specific disability concerns that should be taken into account. As physically demanding as climbing is, there is still a portion of the climbing community who are paraclimbers. There is even an International Federation of Sport Climbing world championship for Paraclimbers. \cite{rcc_climbing_nodate} The training needs of these climbers should be addressed by the app. An example of a possible solution to this problem is incorporating alternative assisted exercises that accommodate specific needs. This issue extends beyond the app though because while some climbing gyms have accommodations for people with disabilities, many climbing gyms lack disability support. This can be bypassed by providing modified exercises that can be done at home or in a typical commercial gym. \cite{rcc_climbing_nodate}



    
\section{Proposed Timeline}

    May 15: General brainstorming about app contents and presentation to user.
    June 1: General brainstorming about app contents and presentation to user.
    June 15: General brainstorming about app contents and presentation to user.
    July 1: Collect more exercises to add to the app.
    July 15: Collect more exercises to add to the app.
    
    August 1: Begin working on the foundation of the project in React
    
    August 15: Have a working (probably empty) react application
    September 1: Have a partially populated React application so that the first round of user Interviews can be conducted. Also will try to have a comprehensive plan of features yet to be added to present to interviewees.
    
    September 15: Continue populating and designing the app. Take in information from user interviews and adjust project trajectory accordingly.
    
    October 1: Continue to work on things brought up in user interviews. Should have all exercises that will be present in the app by this point. 
    October 15: Continue to work on smoothing out the app. Work on planning second round of user interviews. 
    November 1: Second round of user interviews should be made by now
    November 15: Conduct any necessary (hopefully minimal) last minute tweaks to the project based on user interviews. Analyze results from interviews and ensure that 
    December 1: Project should essentially be finished by now, any final last minute tweaks can be conducted. 
    December 15: Project is finished 
    
\section{Comps Topic Proposal}

\subsection{Problem Context}

\subsection{Technical Background}

\subsection{Prior Work}	

\subsection{Evaluation and Methods}

\subsection{Ethical Considerations}



\printbibliography

\end{document}
