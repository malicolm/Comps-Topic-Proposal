\documentclass[10pt,twocolumn]{article} 

\usepackage{oxycomps} % use the main oxycomps style file


\pdfinfo{
    /Title (Comps Topic Proposal)
    /Author (Malcolm Holman)
}

\title{Comps Topic Proposal}

\author{Malcolm Holman}
\affiliation{Occidental College}
\email{holmanm@oxy.edu}

\begin{document}

\maketitle

\section{Problem Context}

    The problem at hand is how nuanced training for indoor and outdoor rock climbing can be for different people. As a rock climber, I have faced frustrating plateaus that required more training to overcome than I knew how to implement at the time. Today, I have more knowledge about the basics, but I continue to train and would benefit from learning about new exercises This lack of knowledge is common for many beginner (and intermediate) climbers and is the first part of the problem I am trying to address. 

	Being a beginner rock climber is a rewarding experience because most low grade, easier climbs can eventually be worked out with decent strength and poor technique. This stage is often short lived as people begin to want to climb harder problems/routes. Past a certain threshold of difficulty (usually ~V3), technique and strength begin to work together and brute force is not always an option. This is the first place many people begin to get frustrated but with some technique tips, one can get better without much strength training. During this stage, I often found myself watching videos and reading articles trying to piece together tips and tricks from different people across the internet. A climbing app could streamline this process by compiling some standard beginner technique mistakes and how to fix them. These tips can also extend beyond just beginner mistakes to provide intermediate and advanced climbers with more nuanced technique. 

	As climbers enter the intermediate stages, often what prevents people from progressing to harder grades is strength, not technique. While there is always room to improve technique, many intermediate climbers are capable of explaining and understanding the technique needed to do a harder problem but may not have the strength to do so. This problem can directly be solved with climbing specific strength training, and is the second issue addressed by the app. Strength training for climbing is particularly challenging because of how sensitive peoples skin, fingers and finger tendons, wrists, and elbows are to injury and tweaks. As a result, a training plan that works for some people may be too strenuous or aggravate a sensitive spot for others. 
	
	The strength training portion of the app will be focused on workout creation, customization, and user interactions. The create feature would allow users to create their own workouts by choosing from a list of provided exercises as well as uploading custom exercises. User input custom exercises will allow for descriptions, photos, and videos to be uploaded along with them. Many existing climbing training apps can provide exercises and workouts but allow for little customization within the workouts given. This can be solved by having every provided workout allow for each exercise to be swapped individually with prompted alternative exercises that target the same muscle group or an exercise of the user’s choosing. Basic user interaction, such as following others and liking workouts, are also going to be implemented because climbing is typically a social activity. This creates opportunities for people to share workouts that they have made with friends and fellow climbers who can then use or modify the workout.
	
\section{Technical Background}

    I have considered algorithmically generating workouts for people based on input criteria but I think that there are ethical concerns associated with the users risk of injury. It is a feature that could be implemented in the future. Without algorithmic workout generation, there is not much technically needed beyond mobile app development. I have worked with web apps in the past and I understand the basic workflow. For this project, since it is a mobile app not a web app, I will have to learn new things. I will be using react native and Firebase to build a mobile app that will be optimized for iPhone.

    
    I chose Firebase because it is a free database service that should be able to handle the amount of reading and writing that will occur while developing my project with the option to pay for more in the future. There is also ample documentation making it easier to pick up. I chose react because it is faster and more streamlined than vanilla JavaScript. 


\section{Prior Work}

    There are more plenty of climbing apps available and even more general fitness/workout apps but none have the same customization as the app being proposed. The most popular climbing training app on Apple’s iPhone app store (my target market) is Crimpd. Crimpd provides users with a handful of useful workouts as well as on-wall circuit training and paid professional training plan. The free workouts that are provided are curated by professional climbers so they are quite good, but to expand upon them you have to pay 13 dollars per month, which is too much. The app also lacks customization of workouts and does not have a timer. 

	The closest apps to what I would like to build are Fitbod and Flex, both of which are general fitness apps, not specifically for climbing. Flex initially prompts the user for their goals and desired intensity and creates an initial workout based upon the input information. The best feature though is what they call “freestyle workouts” which allow the user to select from a large list of workouts that are in their database. The workouts can be broken down into sets, weight used, and duration. Fitbod is another app that I plan to draw inspiration from due to its modular workout design. It allows users to input the equipment that is available to them at their gym and recommends specific workouts accordingly. Following the creation of the workout users have the option to modify the amount of reps, duration, and weight of the exercise. There is also an option to swap the provided exercises for others in their database, similar to Flex.

\section{Methods}


\subsection{Background}
	
	The bulk of the project comes down to the exercises chosen and the information provided to the user about the exercises and their application. Including a wide variety of exercises that can accommodate different skill levels and workout environments is essential for the effectiveness of the project. Essentially, a large collection of selected exercises along with their target muscle groups and brief descriptions will need to be entered into a database for the project to be successful. In addition to presenting users with an abundance of exercises, they must first be given ample information and exercises to prevent any injuries. Additionally, for each exercise, instructions and tips will be provided. The training exercises can be broken down into 3 sub groups: home fitness and strength training, endurance training (in a gym), strength training (in a gym). 
	
\subsection{Injury prevention and Antagonist Muscles}

	(Going to do more research)
	
\subsection{Home training}

	
	There is an entire field of climbing training dedicated to accessible exercises that can be done with little to no additional equipment. Many of these exercises have recently risen in popularity due to the coronavirus pandemic forcing even the most elite rock climbers to train at home. 
	To build finger strength at home, the primary recommendation is the utilization of door frames. A sturdy door frame can serve as both a hangboard or a pull up bar. Door frame edges tend to be on the smaller side, so for all of the following exercises the door frame can be replaced with a larger ledge or a tree branch if necessary to accommodate weaker users. Doorframe finger hangs are trained in the same way as traditional hangboarding, mentioned in the gym strength training section, with the same words of caution: for intermediate and advanced climbers only, requires warm up, use an open crimp, and do not wrap your thumb over. Finger pull ups target both finger and pulling strength and can be done in door frames with the same rules of caution. Hanging leg raises target finger and core strength and can be done on the same door frame with the same level of caution. The goal of leg raises is to dead hang from the frame and bring your legs as high as possible, bending knees to decrease difficulty. 

	(I have additional information from articles with outdoor/open space exercises that will be added)

\subsection{Endurance training}

	For the average gym climber, endurance training may be unnecessary and training attention would be better focused elsewhere. However, this does not mean that everybody cannot benefit from some form of endurance training. Climbing endurance training is different from typical endurance training for activities such as running or swimming. Climbing differs from these sports because it is not constant sustained muscle output, it is bursts or sequences of hard moves followed by some rest. Steve Bechtel, a rock climber and coach with several first ascents worldwide, wrote an article on www.climbstrong.com about endurance training. He breaks the training down into 5 key categories: strength and power, power-endurance, intensive endurance, extensive endurance, and stamina. For the sake of this project, only extensive, intensive, and power endurance will be included.

\subsection{Extensive Endurance}

	Extensive endurance is most useful for longer full pitch routes, trad climbing, and outdoor climbing in general. Extensive endurance training is less useful for indoor boulderers as most indoor bouldering walls do not exceed 15 feet in height. The training is focused on being able to withstand low amounts of muscular stress for long periods of time. The intensity of this training is akin to that of  a warmup, so never reaching a point of failure. The training is intervallic and Bechtel recommends starting with 5 minutes on the wall followed by 5 minutes of rest. As strength and endurance builds, climbers can progress up to 30 minutes on the wall at a time. Because this climbing is easier and requires less effort, it also does not demand as much focus from the climber. This should be avoided because climbing with poor, unfocused technique builds inefficient muscle memory. 

\subsection{Intensive}

	Intensive endurance, according to Bechtel, is the most important area of endurance training to target. It is most useful for hard medium length route climbing but can be applicable to longer boulder problems as well. The difficulty is supposed to feel a step harder than extensive endurance with shorter intervals. The interval splits are shorter and closer together for intensive endurance, Bechtel recommends 9 minutes of continuous climbing with a problem every 90 seconds repeated for multiple sets. The 6 problems should be somewhat difficult but climbers should not be completely exhausted at the end. Climbers should start this training targeting easier climbs while they figure out what climbs fall into this level of difficulty, gradually including harder climbs. Between sets, it is recommended that climbers take 10-15 minutes of rest to ensure that the next set can be executed with fresh muscles.

\subsection{Power Endurance}

	Power endurance is a more complicated form of endurance because it can be broken down into aerobic power and anaerobic capacity. There is more risk associated with it because the increased intensity leads to a larger risk for injury. Generally it is not good to train power endurance more than 2 times per week, any more would require a trainer or a coach to ensure safety. Power endurance training should leave climbers feeling pumped and exhausted by the end of their session.

	Aerobic power training is the more sustained strength of the two, more commonly applied to hard route climbing. A recommended way to train aerobic power is doing 5 sets of 2 reps, with each rep being a double lap of a route. Rest between reps is recommended to be 1 minute, with 10 minutes in between sets to ensure a fresh attempt. The grade of aerobic power training should be 1-2 grades below the climber's consistent onsight grade, which means that by the second rep climbers are pushing their limits. Climbers should be pumped throughout and failing while training aerobic power should happen sometimes. 

	Anaerobic capacity training is focused on high intensity low volume training, more commonly found in bouldering. The recommended split is 6 sets of 3 reps with each rep being a hard, flash level boulder problem. The problems should be hard enough that the moves are not possible to maintain for more than 30-45 seconds. By the end of anaerobic capacity training, climbers should feel powered out and unable to climb anything else. Finally, the chosen problems should be steep and powerful, this training is less effective when climbing 

\section{Evaluation}

	Evaluation of the project is complicated because it is tempting to want to measure its success based upon the results of climbers who train using the provided exercises which is unrealistic for several reasons. First, this would mean that the app would have to be deployed so that users can use it and document their training on their own time. While deployment is realistically possible, it would not occur until late in the project’s timeline leaving little room for progress assessment. Another limitation is the amount of data that would be asked of users without any incentive. Finally, the collection of data from multiple participants over time is too challenging for a team of one to collect in the provided time frame. So, instead of hard qualitative metrics, user interviews will be conducted to assess the effectiveness, design quality, and issues with the app. 

\subsection{Background}

	Interviews can be broken down into 3 distinct types: structured, unstructured, and semi structured. Structured interviews use a fixed series of close ended questions that is the same for every user. Structured interviews are useful because they are quick, allow for more participants, and give more concrete results for analysis. The drawback of structured interviews is that they do not allow for users to elaborate on their ideas. As such, they are more useful towards the end of a project because they can help to assess if goals were accomplished and not much else.  Unstructured interviews lack guided questions and instead flow with the user. They are useful because they accumulate the most possible qualitative data from the users, find pain points and problems, and provide solutions and suggestions. Unstructured interviews need to be carefully conducted because the lack of control over information obtained can create user bias.

\subsection{Interviews}

	For this project, the first round of interviews will be conducted during early stages of production, ideally within the first weeks of the school year. This will allow for the target demographic to have early input about what they would want to see in the app’s final form. Early open ended interviews will also help to identify any glaring holes in the project. These interviews will be less about measuring or assessing the success of the project and more about the direction that it should be headed. 
About three quarters of the way through the semester, ideally sometime in (early November?), a second round of user interviews will be conducted. The second round of interviews will likely be a combination of structured and semi structured interviews. There will be a set of questions that will be asked every time with room for elaboration on those questions. This combination will help to provide late stage feedback which will be useful to assess the success of the project. 
Following the collection of qualitative data from the first and second rounds of interviews, an analysis will take place. Throughout the interview process, ample notes will be taken in addition to a recording of the conversation, provided user consent is obtained. One method of analysis I plan to use is thematic analysis. Thematic analysis consists of: reviewing the notes and the recording, summarizing key repeated points and giving them codes, categorizing the codes, grouping the codes according to similarities, and finally using the categorized groups to find specific themes. These themes could range from criticism of pain points to useful suggestions and will be useful for both the first and second stage of interviews. 


\section{Comps Topic Proposal}

\subsection{Problem Context}

\subsection{Technical Background}

\subsection{Prior Work}	



\end{document}
